\documentclass{article}

\usepackage{amsmath, amssymb}
\usepackage{geometry}
\usepackage{fancyhdr}
\usepackage{siunitx}
\usepackage{circuitikz}
\usepackage{multicol}
\usetikzlibrary{decorations.markings}

\geometry{letterpaper, margin=2.5cm}
%\setlength{\headheight}{15pt}
\pagestyle{fancy}
\fancyhf{}
%\rhead{3CV6}
\lhead{Tarea: Bobinas acopladas magneticamente}
%\cfoot{\thepage}
%\renewcommand{\figurename}{Figura}
\sisetup{per-mode=fraction}

\begin{document}
	\begin{multicols}{2}
	\section{}% circuito 1
		\begin{circuitikz}[american]
			% Nodos
			% nodos de la bobina 1
			\coordinate (A) at (0,0);
			\coordinate (B) at (3,0);
			\coordinate (C) at (3,3);
			\coordinate (D) at (0,3);
			%nodos de la bobina 2
			\coordinate (E) at (5,2);
			\coordinate (F) at (8,2);
			\coordinate (H) at (8,5);
			\coordinate (G) at (5,5);
			%bobina 3
			\coordinate (I) at (5,1);
			\coordinate (J) at (8,1);
			\coordinate (K) at (5,-2);
			\coordinate (L) at (8,-2);
			%bobina 4
			\coordinate (M) at (5,-3); %para vertical
			\coordinate (N) at (5,-6); %para vertical
			\coordinate (O) at (8,-3);
			\coordinate (P) at (8,-6);
			%paridad de bobinas o polaridad, ya no se si es igual
			\coordinate (a) at (3.15,2.22); %paridad de la bobina 1
			\coordinate (e1) at (4.85,4.2); %bobina 2
			\coordinate (i1) at (4.85,-1.2); % bobina 3
			\coordinate (m1) at (4.85,-5.2); % bobina 4

		    % circuito de bobinas
		    \draw
		    %bobina 1
		    (A) to (B)
		    (D) to (C)
		    (1.5,3) to[open,i>^=$i_1$] (C)
		    (C) to[L=$L_1$] (B)
		    %bobina 2
		    (E) to (F) %parte inferior
		    (E) to[L=$L_2$] (G) % en medio
		    (G) to (H) % parte superior
		    (H) to[open,i>^=$i_2$] (3,5)
		    %bobina3
		    (K) to[L=$L_3$] (I)
		    (I) to (J)
		    (K) to (L)
		    (J) to[open,i>^=$i_3$] (3,1)
		    %bobina 4
		    (N) to[L=$L_{4}$] (M)
		    (N) -- (P)
		    (M) -- (O)
		    (O) to[open,i>^=$i_4$] (M);
		    % Puntos visibles en los nodos
		    \draw
		        (a) node[circle, fill=black, inner sep=1.2pt] {}
		        (e1) node[circle, fill=black, inner sep=1.2pt] {}
		        (i1) node[circle, fill=black, inner sep=1.2pt] {}
		        (m1) node[circle, fill=black, inner sep=1.2pt] {}
		        [thick, -] (3.2,2.4) .. controls (3.5,3.7) .. (4.7,4.3)
		        [thick, -] (4.7,0.3) .. controls (3.6,1.4) .. (4.7,2.7)
		        [thick, -] (3.25,0.8) .. controls (3.6,-0.2) .. (4.7,-1)
		        [thick, -] (3.15,0.6) .. controls (3.6,-2) .. (4.7,-4.5)
		        [thick, -] (4.7,-1.2) .. controls (4,-2.7) .. (4.7,-3.7)
		        [thick, -] (5.4,3.5) .. controls (6,-0.5) .. (5.4,-4.5);
		    \draw[red, thick, <->] (2.6,-0.5) .. controls (3.5,-0.3) .. (3.7,-0.2);
		    \draw[red, thick, <->] (6.5,1.35) .. controls (6,0.9) .. (5.9,0.25);
		    \node at (1.5,1.5){$L_1=\SI{6}{\henry}$};
		    \node at (4.8,1.4){$L_{23}=\SI{1.5}{\henry}$};
		    \node at (6.5,3.5){$L_{2}=\SI{8}{\henry}$};
		    \node at (6.5,-0.5){$L_{3}=\SI{4}{\henry}$};
		    \node at (1.7,-0.5){$L_{13}=\SI{2}{\henry}$};
		    \node at (3,4.3){$L_{12}=\SI{3}{\henry}$};
		    \node at (6.5,-4.5){$L_{4}=\SI{3}{\henry}$};
		    \node at (3.2,-3.5){$L_{14}=\SI{1}{\henry}$};
		    \node at (5,-2.5){$L_{34}=\SI{1}{\henry}$};
		    \node at (7.5,1.4){$L_{24}=\SI{1.5}{\henry}$};

		\end{circuitikz}
	\end{multicols}
		\newpage
	\begin{multicols}{2}
	\section{} % circuito 2
		\begin{circuitikz}[american]
			% Nodos
			\coordinate (A) at (0,0);
			\coordinate (a) at (3.15,0.8); %bobina 1, polaridad
			\coordinate (B) at (3,0);
			\coordinate (C) at (3,3);
			\coordinate (D) at (0,3);
			\coordinate (E) at (5,1);
			\coordinate (F) at (5,2);
			\coordinate (F1) at (5,5);
			\coordinate (G) at (8,1);
			\coordinate (H) at (5,-2);
			\coordinate (G2) at (8,-2);
			\coordinate (G3) at (8,2);
			\coordinate (G4) at (8,5);
			\coordinate (e1) at (4.85,2.8); %bobina 2, polaridad
			\coordinate (h1) at (4.85,0.25); %bobina 3, polaridad

		    % circuito de bobinas
		    \draw
		    (A) to (B)
		    (D) to (C)
		    (1.5,3) to[open,i>^=$i_1$] (C)
		    (C) to[L=$L_1$] (B)
		    (H) to[L=$L_3$] (E)
		    (E) to (G)
		    (H) to (G2)
		    (F) to (G3)
		    (F) to[L=$L_2$] (F1)
		    (F1) to (G4)
		    (8,2) to[open,i>^=$i_2$] (3,2)
		    (G) to[open,i>^=$i_3$] (3,1);
	% 1 0 -1 -2
		    % Puntos visibles en los nodos
		    \draw
		        (a) node[circle, fill=black, inner sep=1.2pt] {}
		        (e1) node[circle, fill=black, inner sep=1.2pt] {}
		        (h1) node[circle, fill=black, inner sep=1.2pt] {}
		        [thick, -] (3.2,2.4) .. controls (3.5,3.7) .. (4.7,4.3)
		        [thick, -] (4.7,0.3) .. controls (3.6,1.4) .. (4.7,2.7)
		        [thick, -] (3.2,0.6) .. controls (3.4,-0.5) .. (4.7,-1);
		    \node at (1.5,1.5){$L_1=\SI{6}{\henry}$};
		    \node at (5,1.4){$L_{23}=\SI{0.7}{\henry}$};
		    \node at (6.5,3.5){$L_{2}=\SI{5}{\henry}$};
		    \node at (6.5,-0.5){$L_{3}=\SI{3}{\henry}$};
		    \node at (2.7,-0.5){$L_{13}=\SI{0.6}{\henry}$};
		    \node at (3,4.3){$L_{12}=\SI{0.8}{\henry}$};
		\end{circuitikz}
		Obtener:
		\begin{enumerate}
			\item El valor de las inductancias mutuas ($\pm L_{kl}$) 
			\item El valor de las invertancias propias y mutuas ($\Gamma_{kl}$)
			\item El sistema de ecuaciones de corriente para cada bobina en función de los voltajes
			\item El sistema de ecuaciones de caída de voltaje en cada bobina en funcion de las corrientes
		\end{enumerate}
	\end{multicols}
	\newpage
	\begin{multicols}{2}
		\section{} % circuito 3
		\begin{circuitikz}[american]
			% Nodos
			\coordinate (A) at (0,0);
			\coordinate (a) at (3.15,2.22);
			\coordinate (B) at (3,0);
			\coordinate (C) at (3,3);
			\coordinate (D) at (0,3);
			\coordinate (E) at (5,1);
			\coordinate (F) at (5,2);
			\coordinate (F1) at (5,5);
			\coordinate (G) at (8,1);
			\coordinate (H) at (5,-2);
			\coordinate (G2) at (8,-2);
			\coordinate (G3) at (8,2);
			\coordinate (G4) at (8,5);
			\coordinate (e1) at (4.85,2.8); %bobina 2, polaridad
			\coordinate (h1) at (4.85,0.25); %bobina 3, polaridad

		    % circuito de bobinas
		    \draw
		    (A) to (B)
		    (D) to (C)
		    (1.5,3) to[open,i>^={\(i_{1}=2\sin(500t)\SI{}{\ampere}\)}] (C)
		    (C) to[L=$L_1$] (B)
		    (H) to[L=$L_3$] (E)
		    (E) to (G)
		    (H) to (G2)
		    (F) to (G3)
		    (F) to[L=$L_2$] (F1)
		    (F1) to (G4)
		    (G4) to[open,i>^={\(i_{2}=1.5\sin(500t)\SI{}{\ampere}\)}] (3,5)
		    (G) to[open,i>^={\(i_{3}=0.5\sin(500t)\SI{}{\ampere}\)}] (3,1);
	% 1 0 -1 -2
		    % Puntos visibles en los nodos
		    \draw
		        (a) node[circle, fill=black, inner sep=1.2pt] {}
		        (e1) node[circle, fill=black, inner sep=1.2pt] {}
		        (h1) node[circle, fill=black, inner sep=1.2pt] {}
		        [thick, -] (3.2,2.4) .. controls (3.5,3.7) .. (4.7,4.3)
		        [thick, -] (4.7,0.3) .. controls (3.6,1.4) .. (4.7,2.7)
		        [thick, -] (3.2,0.6) .. controls (3.4,-0.5) .. (4.7,-1);
		    \node at (1.5,1.5){$L_1=\SI{80}{\milli\henry}$};
		    \node at (5,1.4){$L_{23}=\SI{10}{\milli\henry}$};
		    \node at (6.5,3.5){$L_{2}=\SI{50}{\milli\henry}$};
		    \node at (6.5,-0.5){$L_{3}=\SI{30}{\milli\henry}$};
		    \node at (2.2,-0.5){$L_{13}=\SI{15}{\milli\henry}$};
		    \node at (3,4.3){$L_{12}=\SI{20}{\milli\henry}$};
		\end{circuitikz}
		Obtener:
		\begin{enumerate}
			\item Calcular la caída de tensión de cada bobina
		\end{enumerate}	
	\end{multicols}
	\newpage
	\begin{multicols}{2}
		\section{} % circuito 3
		\begin{circuitikz}[american]
			% Nodos
			\coordinate (A) at (0,0);
			\coordinate (B) at (3,0);
			\coordinate (C) at (3,3);
			\coordinate (D) at (0,3);
			%4 nodos para bobina 1
			\coordinate (E) at (5,0);
			\coordinate (F) at (5,3);
			\coordinate (G) at (8,0);
			\coordinate (H) at (8,3);
			%4 nodos para bobina 2
			\coordinate (a) at (3.15,0.8); % bobina 1, paridad
			\coordinate (e1) at (4.85,2.22); %bobina 2, polaridad

		    % circuito de bobinas
		    \draw
		    (A) to (B) %parte inferior
		    (D) to (C)
		    (D) to[open,i>^=$i_{1}$] (C)
		    (C) to[L=$L_1$] (B)
		    %bobina 1
		    (E) -- (G) %parte inferior
		    (F) -- (H)
		    (E) to[L=$L_2$] (F)
		    (F) to[open,i<^=$i_{2}$] (H);
		    %bobina 2
		    
		    \draw[red, thick, postaction={decorate}, decoration={markings, mark=at position 0.5 with {\arrow{>}}}] (2.6,2.8) -- (2.6,0.2) node[midway,left]{$V_1$}; % sentido de la tension a la bobina 1

		    \draw[red, thick, postaction={decorate}, decoration={markings, mark=at position 0.5 with {\arrow{>}}}] (5.4,2.8) -- (5.4,0.2) node[midway,right]{$V_2$}; % sentido de la tension a la bobina 2

		    % Puntos visibles en los nodos
		    \draw
		        (a) node[circle, fill=black, inner sep=1.2pt] {}
		        (e1) node[circle, fill=black, inner sep=1.2pt] {}
		        [thick, -] (3.2,2.4) .. controls (4,3.2) .. (4.8,2.4);
		    \node at (1,1.5){$L_1=\SI{150}{\milli\henry}$};
		    \node at (7,1.5){$L_{2}=\SI{75}{\milli\henry}$};
		    \node at (4,4){$L_{12}=\SI{20}{\milli\henry}$};
		\end{circuitikz}
		Obtener:
		\begin{enumerate}
			\item Calcular la caída de tensión de cada bobina
			cuando:
			\begin{enumerate}
				\item $V_1 = 2.5\sin(1000t) \SI{}{\volt}$
				\item $V_2 = 0.75\sin(1000t) \SI{}{\volt}$
			\end{enumerate}
		\end{enumerate}	
	\end{multicols}
			
\end{document}